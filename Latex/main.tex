\documentclass[a4paper]{article}
\usepackage[utf8]{inputenc}
\usepackage[slovene]{babel}
\usepackage{graphicx}
\usepackage{hyperref}
\usepackage[nottoc]{tocbibind}
\usepackage{caption}
\usepackage{subcaption}
\usepackage{amsmath}
\usepackage{ dsfont }
\usepackage{siunitx}
\usepackage{multimedia}
\usepackage[table,xcdraw]{xcolor}
\setlength\parindent{0pt}

\newcommand{\ddd}{\mathrm{d}}
\newcommand\myworries[1]{\textcolor{red}{#1}}
\newcommand{\Dd}[3][{}]{\frac{\ddd^{#1} #2}{\ddd #3^{#1}}}

\begin{document}
\begin{titlepage}
    \begin{center}
        \includegraphics[]{logo.png}
        \vspace*{3cm}
        
        \Huge
        \textbf{Numerična minimizacija}
        
        \vspace{0.5cm}
        \large
        3. naloga pri Modelski Analizi 1

        \vspace{4.5cm}
        
        \textbf{Avtor:} Marko Urbanč (28232019)\ \\
        \textbf{Predavatelj:} prof. dr. Simon Širca\ \\
        \textbf{Asistent:} doc. dr. Miha Mihovilovič\ \\
        
        \vspace{2.8cm}
        
        \large
        24.2.2023
    \end{center}
\end{titlepage}
\tableofcontents
\newpage
\section{Uvod}
Numerično minimizacijo poznamo tudi pod širšim imenom matematična optimizacija.
V zelo preprostih pojmih gre za izbito najbolj primernega elementa iz neke množice, 
glede na podane kriterije. Običajno je ta množica neka funkcija, ki jo želimo minimizirati.
Kriterij pa je podan z neko funkcijo, ki nam pove kako dober je neki element. Če se to sliši
zelo podobno kot uvod pri prejšnji nalogi, kjer smo si pogledali Linearno programiranje, je to
zato, ker je to v bistvu ista stvar. Razlika je le v tem, da je pri linearnem programiranju
funkcija, ki jo minimiziramo linearna, pri numerični minimizaciji pa je ta funkcija lahko
poljubna. \\

Z največjim veseljem bi napisal še kakšen bolj matematičen uvod, ampak to bi pomenilo, da bi 
se moral spustiti v podrobnosti delovanja posameznih optimizacijskih algoritmov, kar pa je
precej obsežna tema, še sploh če bi se želel dotakniti vseh, ki sem jih poskusil, ker jih je
res veliko. \\

Veliko problemov se prevede na optimizacijske probleme, zato je to zelo pomembna tema. Če 
ne drugega, je dandanes strašno priljubljeno strojno učenje, ki je v bistvu nič drugega kot
optimizacija neke funkcije, ki nam pove kako dobro se nek model prilega podatkom. Tu se pojavi 
poanta, ki sem jo želel (a neuspešno) povedati v prejšnji nalogi. Optimizacija funkcije z npr. 40 
parametri je zelo malo. V praksi smo zmožni optimizirati funkcije z več milijoni parametrov in tu
se vmeša meni priljubljen High Performance Computing in to da sem želel pri prejšnji nalogi 400k
dimenzij. Tukaj sicer ne bomo počeli tega, ne ker ne bi želel, ampak ker žal nisem utegnil. \\

\section{Naloga}

\section{Opis reševanja}

\section{Rezultati}


\section{Komentarji in izboljšave}

\newpage
\bibliographystyle{unsrt}
\bibliography{sources}
\end{document}
