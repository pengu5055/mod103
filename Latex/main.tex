\documentclass[a4paper]{article}
\usepackage[utf8]{inputenc}
\usepackage[slovene]{babel}
\usepackage{graphicx}
\usepackage{hyperref}
\usepackage[nottoc]{tocbibind}
\usepackage{caption}
\usepackage{subcaption}
\usepackage{amsmath}
\usepackage{ dsfont }
\usepackage{siunitx}
\usepackage{multimedia}
\usepackage[table,xcdraw]{xcolor}
\setlength\parindent{0pt}

\newcommand{\ddd}{\mathrm{d}}
\newcommand\myworries[1]{\textcolor{red}{#1}}
\newcommand{\Dd}[3][{}]{\frac{\ddd^{#1} #2}{\ddd #3^{#1}}}

\begin{document}
\begin{titlepage}
    \begin{center}
        \includegraphics[]{logo.png}
        \vspace*{3cm}
        
        \Huge
        \textbf{Numerična minimizacija}
        
        \vspace{0.5cm}
        \large
        3. naloga pri Modelski Analizi 1

        \vspace{4.5cm}
        
        \textbf{Avtor:} Marko Urbanč (28232019)\ \\
        \textbf{Predavatelj:} prof. dr. Simon Širca\ \\
        \textbf{Asistent:} doc. dr. Miha Mihovilovič\ \\
        
        \vspace{2.8cm}
        
        \large
        24.2.2023
    \end{center}
\end{titlepage}
\tableofcontents
\newpage
\section{Uvod}
Numerično minimizacijo poznamo tudi pod širšim imenom matematična optimizacija.
V zelo preprostih pojmih gre z

\section{Naloga}

\section{Opis reševanja}

\section{Rezultati}


\section{Komentarji in izboljšave}

\newpage
\bibliographystyle{unsrt}
\bibliography{sources}
\end{document}
